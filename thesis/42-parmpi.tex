\chapter{Реализация параллельной генерации состояний}
\label{cha:parmpi}

\section{Выбор средств распараллеливания}
\label{sec:paral-selection}

\section{Взаимодействие узлов}
\label{sec:mpi-interaction}

\subsection{Сравнение примитивов обмена сообщениями в MPI}
\label{sec:mpi-primitives}

\subsection{Схема асинхронного обмена сообщениями}
\label{sec:async-mpi-queue}

% \begin{figure}[ht]
%   \centering
%   \includegraphics[width=1.0\textwidth]{../graphics/mpi-async-seq}  
%   \caption{Асинхронное взаимодействие узлов (MPI)}
%   \label{fig:mpi-async-seq}
% \end{figure}

\section{Распределенное завершение}
\label{sec:distributed-termination}

\subsection{Алгоритм Дейкстры}
\label{sec:distr-term-dijkstra}

\begin{figure}[ht]
  \centering
  \includegraphics[width=1\textwidth]{../graphics/distr-termination}  
  \caption{Распределенное завершение}
\label{fig:dist-term}
\end{figure}

%%% Local Variables: 
%%% mode: latex
%%% TeX-master: "main"
%%% End: 
