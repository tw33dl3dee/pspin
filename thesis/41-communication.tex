\chapter{Взаимодействие узлов}
\label{cha:communication}

\section{Выбор средств параллельного выполнения}
\label{sec:parall-selection}

Используемое для параллельного выполнения средство должно обладать следующими свойсвами:
\begin{enumerate}
\item поддерживать модель неразделяемой памяти (системы с разделяемой памятью не
  представляют в данном случае интереса, поскольку основной предпосылкой к
  распараллеливанию является нехватка памяти на отдельной машине);
\item быть достаточно универсальным и распространнеым.
\end{enumerate}

Для данной задачи подходят два средства -- MPI (Message Passing Interface, \cite{MPI}) и UPC (Unified
Parallel Compiler,~\cite{UPC12}).

\section{Огранизация пересылки состояний}
\label{sec:mpi-interaction}

\subsection{Сравнение способов и примитивов взаимодействия в MPI}
\label{sec:mpi-primitives}

\subsection{Схема асинхронного обмена сообщениями}
\label{sec:async-mpi-queue}

\begin{figure}[ht]
  \centering
  \includegraphics[width=1.0\textwidth]{../graphics/mpi-async-seq}  
  \caption{Асинхронное взаимодействие узлов (MPI)}
  \label{fig:mpi-async-seq}
\end{figure}

\subsection{Формат сообщений}
\label{sec:message-format}

\section{Локальная и сетевая очереди состояний}
\label{sec:local-network-queue}

\section{Распределенное завершение}
\label{sec:distributed-termination}

\subsection{Алгоритм Дейкстры}
\label{sec:distr-term-dijkstra}

\begin{figure}[ht]
  \centering
  \includegraphics[width=1\textwidth]{../graphics/distr-termination}  
  \caption{Распределенное завершение}
\label{fig:dist-term}
\end{figure}

\section{Журналирование и отладка}
\label{sec:mpi-logging}


%%% Local Variables: 
%%% mode: latex
%%% TeX-master: "thesis"
%%% End: 
