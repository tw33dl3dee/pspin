\chapter{Взаимодействие узлов}
\label{cha:communication}

\section{Выбор средств параллельного выполнения}
\label{sec:parall-selection}

Используемое для параллельного выполнения средство должно обладать следующими свойсвами:
\begin{enumerate}
\item поддерживать модель неразделяемой памяти (системы с разделяемой памятью не
  представляют в данном случае интереса, поскольку основной предпосылкой к
  распараллеливанию является нехватка памяти на отдельной машине);
\item быть достаточно универсальным и распространнеым.
\end{enumerate}

По первому условию подходят три средства~--- MPI (Message Passing Interface), PVM (Parallel
Virtual Machine,~\cite{PVM}) и UPC (Unified Parallel Compiler,~\cite{UPC12}).

MPI является де-факто стандартом написания параллельных приложения для
высокопроизводительных систем и кластеров~\cite{MPI} благодаря поддержке множества языков
(C, C++, Fortran, Java, Python\etc), аппаратных платформ и коммуникационного оборудование
(Ethernet, Infiniband, iWARP\etc). Кроме того, на большинстве кластеров предустановлено
соответствующее ПО для его использования. Исходя из этих преимуществ, в качестве средства
параллельного выполнения был выбран MPI.

\section{Огранизация пересылки состояний}
\label{sec:mpi-interaction}

\paragraph{Сравнение способов и примитивов взаимодействия в MPI}
\label{sec:mpi-primitives}

MPI представляет собой фактически наборов примитивов для обмена сообщениями между
процессами, параллельно выполняющимися на разных узлах с неразделяемой памятью, и для
групповых операций, таких, как синхронизация.

MPI предоставляет следующий набор примитивов взаимодействия.

\begin{enumerate}
\item Синхронная передача сообщений (\Code{Ssend}, \Code{Srecv}). Делающий вызов процесс
  ожидает, пока сообщение не будет доставлено \emph{и} прочитано получателем.
\item Блокирующая передача сообщений (\Code{Send}, \Code{Recv}). Делающий вызов процесс
  ожидает, пока сообщение будет отправлено получателю (при этом его доставка не
  имеет значения).
\item Асинхронная передача сообщений (\Code{Isend}, \Code{Irecv}). Управление возвращается
  процессу немедленно, передача происходит в фоновом режиме.
\item Активный удаленный доступ к памяти (Active RMA, \Code{Win_start},
  \Code{Win_complete}, \Code{Win_post}, \Code{Win_wait}). Каждый процесс создает
  \Term{окно}~--- раздел памяти для всеобщего доступа, при этом один процесс может
  обращаться к окнам других процессов при помощи \Code{Put} и \Code{Get}.
\item Пассивный удаленный доступ к памяти (Passive RMA, \Code{Win_lock},
  \Code{Win_unlock}). Аналогично active RMA, но со принимающего данные процесса не
  требуется какого либо участия.
\end{enumerate}

Синхронная передача очевидно не подходит, поскольку ожидание доставки будет лишней
задержкой в работе, так что из первых трех методов оптимальной является асинхронная
передача.

Наиболее привлекательным выглядит механизм passive RMA, поскольку от принимающего процесса
не требуется каких-либо действий по приему сообщения в очередь, а именно такому поведению
соответствует описанный в разделе~\ref{sec:distr-generation} алгоритм. Однако, RMA, хоть и
является более простым в использовании, на современных реализациях MPI работает медленнее,
чем остальные примитивы~--- это связано с сложностью синхронизации доступа к памяти окнаи
с организацией приема данных на пассивной стороне. На рис.~\ref{fig:mpi-primitives}
показано сравнение их производительности для реализации Intel MPICH.

Для получения этого сравнения была использована утилита mpibench, которая имитирует
передачу сообщений определенного размера между заданным числом узлов в выбранном режиме
(по кольцу, от каждого ко всем\etc).

\begin{figure}[ht]
  \centering
  %\includegraphics[width=0.5\textwidth]{../graphics/mpi-primitives}
  \caption{Сравнение производительности примитивов MPI}
  \label{fig:mpi-primitives}
\end{figure}

Из сравнения видно, что RMA почти в 4 раза медленнее, чем асинхронная передача
сообщений. Поскольку ожидается, что задержки на передачу сообщений будут составлять
большую часть времени генерации, в качестве используемого примитива была выбрана
асинхронная передача-прием сообщений, несмотря на удобство RMA.

\paragraph{Схема асинхронного обмена сообщениями}
\label{sec:async-mpi-queue}

После того, как буфер с данными отправлен вызовом \Code{Isend}, его нельзя вторично
использовать, пока сообщение не будет доставлено~--- необходимо дождаться завершения
операции доставки. Поэтому используется набор из $Q_1$ предвыделенных буферов (оптимальное
значение $Q_1$ необходимо выбирать, исходя из параметров кластера, скорости работы
сети\etc).

Отправка сообщения происходит следующим образом:
\begin{enumerate}
\item выбирается первый свободный буфер (не помеченный как участвующий в асинхронной
  операции);
\item если такой буфер не найден~--- делается вызов \Code{Waitany} по всему набору
  буферов, который возвращает управление, как только хотя бы одна из операций завершится;
\item операции по некоторым буферам уже могли завершиться к этому моменту~--- в этом
  случае \Code{Waitany} возвращает управление немедленно, помечая первый из таких буферов
  как свободный; при этом может больше одной завершившейся операции, и, чтобы избежать
  повторных вызовов \Code{Waitany} при посылке следующих сообщений, \Code{Testany}
  вызывается в цикле, пока все такие буфера не будут помечены освобожденными;
\item передавемое состояние копируется в буфер;
\item делается вызов \Code{Isend}, запускающий асинхронную передачу, и буфер помечается
  используемым.
\end{enumerate}

Описанный алгоритм отправки изображен в виде блок-схемы на
рис.~\ref{fig:mpi-send-flowchart}.

\begin{figure}[!tb]
  \centering
  \includegraphics[height=0.8\textheight]{../graphics/mpi-send-flowchart}
  \caption{Блок-схема алгоритма отправки сообщения}
  \label{fig:mpi-send-flowchart}
\end{figure}

Для приема сообщений используется другой набор из $Q_2$ буферов (для простоты используется
одно и то же $Q_1 = Q_2 = Q$). Каждый из них предварительно передается в вызов
\Code{Irecv}, таким образом, запускается сразу $Q_2$ операций асинхронного приема. Прием
сообщения делается следующим образом:
\begin{enumerate}
\item делается вызов \Code{Waitany}, возвращающий управление, когда хотя бы одна из
  операций приема завершится;
\item по аналогии с приемом, \Code{Waitany} может вернуть управление сразу;
\item производится работа с принятым буфером;
\item когда буфер больше не нужен, делается вызов \Code{Irecv}, запускающий очередную
  операцию приема.
\end{enumerate}

Возможен также вариант приема сообщения без ожидания, когда вместо \Code{Waitany}
используется \Code{Testany}, проверяющий, нет ли уже завершенных операций
приема. Подробнее об использовании обоих вариантов говорится далее в
разделе~\ref{sec:local-network-queue}.

Наборы буферов приема-отправки выделены в отдельные высокоуровневые примитивы (асинхронные
очереди MPI) с отдельными функциями для работы с ними.

Описанная схема асинхронного взаимодействия показана в виде диаграммы последовательностей
на рис.~\ref{fig:mpi-async-seq}. Здесь \Code{P1} и \Code{P2}~--- два взаимодействующих
процесса на разных узлах, один показан в роли отправителя, другой~--- в роли
приемника. \Code{C1} и \Code{C2}~--- их MPI-коммуникаторы, а \Code{Q1} и \Code{Q2}
представляют собой описанные выше асинхронные очереди.

\begin{figure}[ht]
  \centering
  \includegraphics[width=1.1\textwidth]{../graphics/mpi-async-seq}  
  \caption{Асинхронное взаимодействие узлов (MPI)}
  \label{fig:mpi-async-seq}
\end{figure}

\section{Локальная и сетевая очереди состояний}
\label{sec:local-network-queue}

В описании приема сообщений в разделе~\ref{sec:async-mpi-queue} не сказано, в какие именно
моменты происходит прием MPI-сообщений. В алгоритме, приведенном в
разделе~\ref{sec:distr-generation}, есть операция выборки следующего состояния из очереди,
\Code{Queue $\rightarrow$ state}.

В реальности, у какждого узла фактически есть две очереди: локальная очередь
\Code{Queue$_L$}, куда делается вставка новых локально сгенерированных состояний, и
неявная сетевая очередь \Code{Queue$_N$}, представленная набором MPI-буферов (неявная
потому, что какой-либо порядок добавления--выборки в ней отсутствует).

При отправке состояния другому узлу оно вставляется в его сетевую очередь, а при выборке
надо использовать обе очереди. Можно предложить два варианта опроса сетевой очереди:
\begin{enumerate}
\item всегда сначала выбирать состояние из локальной очереди; если локальная очередь
  закончилась, ожидать появления состояний в сетевой и выбирать из нее;
\item проверять сначала сетевую очередь на предмет наличия в ней сообщения; если их нет,
  проверять локальную; если обе пусты, ожидать появления сообщений в сетевой.
\end{enumerate}

Оба метода имеют свои достоинства и недостатки.
\begin{itemize}
\item При небольшом количестве узлов локальная очередь состояний на отдельном узле может
  довольно долго не опустевать~--- если каждое состояние порождает несколько новых, и хотя
  бы одно из них принадлежит текущему узлу. Тогда, если сначала опрашивается локальная
  очередь, сообщения подолгу не будут приниматься этим узлом, а у других узлов, по мере
  посылки ему сообщений, будут скапливаться незавершенные операции.
\item Прием сообщений сначала из \Code{Queue$_N$} может приводить к большому росту
  \Code{Queue$_L$}. При полном хэшировании (см.~раздел~\ref{sec:fullhash-store}) размер
  \Code{Queue$_L$} не играет значения: \Code{Queue$_L$} и \Code{Visited} хранятся в одной
  области памяти, поэтому порядок добавления в \Code{Queue$_L$} ничего не меняет: все
  добавленные в нее состояния рано или поздно будут перенесены в \Code{Visited}. Однако
  при битовом хэшировании (см.~раздел~\ref{sec:bithash-store}) состояния не хранятся нигде
  вне обоих очередей, и под \Code{Queue$_L$} отводится отдельная область памяти,
  увеличение размера которой приводит к уменьшению размера хэш-таблицы.
\end{itemize}

Для того, чтобы минимизировать сетевые задержки, предпочтение отдается второму варианту:
сначала проверяется сетевая очередь. Псевдокод алгоритма, с приведенными уточнениями,
имеет вид:

\begin{lstlisting}[style=pseudocode]
Visited = ()
QueueL  = ()
QueueN  = ()

def NextState():
    if not empty(QueueN):
        QueueN -> NextState
    elif not empty(QueueL):
        QueueL -> NextState
    else:
        wait(QueueN)
        QueueN -> NextState

def ParStateSpaceBFS():
    state = initial_state
    do:
        node = StateNode(state)
        if NodeId = node:
            if not state in Visited:
                Visited <- state
                for each new_state in Next(state):
                    QueueL <- new_state
        else:
            node.QueueN <- state
        state = NextState()

ParStateSpaceBFS()
\end{lstlisting}

Возможны две схемы использования состояния, взятых из сетевой очереди \Code{Queue$_N$}:
скопировать их в отдельную область памяти (например, в \Code{Queue$_L$}), после чего сразу
отдать буфер обратно в асинхронную очередь, либо использовать состояние прямо в буфере и
отдать его после завершение обработки состояния. 

При полном хэшировании состояний вторая схема неприменима: в этом случае в хэш-таблицу
будет добавлен указатель на состояние в MPI-буфере, а не в \Code{Visited}, куда состояние
будет перенесено после обработки. Поэтому при полном хэшировании всегда используется
первая схема, тогда как при битовом можно выбрать любую из двух (одна обеспечивает меньший
размер \Code{Queue$_L$}, другая~--- меньшие сетевые задержкие).

\section{Распределенное завершение}
\label{sec:distributed-termination}

Приведенный выше алгоритм не содержит в себе никакого условия завершения. Задача
обнаружения завершения в параллельных расчетах без разделяемой памяти является
нетривиальной: в условиях отсутствия средств атомарного опроса всех остальных узлов
необходим алгоритм, который бы позволил хотя бы одному узлу определить, что вычисления
завершились на всех узлах \emph{и} никаких отправленных, но еще не принятых сообщений в
системе нет.

На рис.~\ref{fig:termination-problem} показан пример взаимодействия узлов, развернутого во
времени. Жирные линии означают, что узел занимается обработкой, стрелки~--- передаваемые
сообщения. Из него видно, что простого опроса узлов, заняты ли они вычислениями,
недостаточно: в момент времени $T_1$ все узлы простаивают, однако имеется непринятое
сообщение, отправленное узлом 3. Настоящим моментом завершения является момент $T_2$,
когда все узлы простаивают \emph{и} все сообщения достигли назначения.

\begin{figure}[htb]
  \centering
  \includegraphics[width=1\textwidth]{../graphics/termination-problem}  
  \caption{Демонстрация задачи о распределенном завершении}
\label{fig:termination-problem}
\end{figure}

\paragraph{Наивное решение}
\label{sec:distr-term-dijkstra}

Наивное решение задачи заключается в том, чтобы ввести на каждом узле $i$ счетчик $C_i$
сообщений, находящихся в пути. Изначально во всех узлах этот счетчик равен $0$, при
отправке сообщения другому узлу он увеличивается на $1$, при приеме~--- уменьшается на
$1$. Сумма счетчиков всех узлов $\sum_iC_i$ в выбранный момент времени есть число
находящихся в пути сообщений. Для подсчета этой суммы на каждом узле вводится
дополнительный счетчик-аккумулятор $A_i$. Первый узел инициализирует его значением $C_i$ и
передает первому узлу. Когда первый узел завершает всю локальную обработку и входит в
состояние простое, он прибавляет к аккумулятору $C_1$ и передает второму\etc. Последний
узел, завершив работу, передает обратно первому $A_N = \sum_iC_i$ и, если это значение
равно нулю, первый узел делает вывод, что сообщений в пути нет и оповещает всех остальных
о завершении. В противном случае ($A_N \neq 0$), первый узел отправляет второму новый
счетчик-аккумулятор, и процесс повторяется заново.

Такой <<наивный>> подход неправильно работает при сценарии, показанном на
рис.~\ref{fig:termination-naive}. Контрольные сообщения, содержащие счетчик-аккумулятор,
показаны курсивом, рядом с ними показано значение этого счетчика. Поскольку узел 3
отправил состояние, которое еще не было получено узлом 2 в момент <<прохождения>> через
него аккумулятора, а узел 4 успел принять сообщение от узла 2 до приема аккумулятора,
суммарное число сообщений выходит нулевым. Сообщения, посланные узлу 2 от узла 3 и узлу 4
от узла 2, не вошли в этот аккумулятор, в результате чего завершение будет обнаружено
узлом 1 неверно~--- узел 2, получивший новое сообщение уже после отправки аккумулятора,
может продолжать его обработку в момент объявления завершения.

\begin{figure}[htb]
  \centering
  \includegraphics[width=1\textwidth]{../graphics/termination-naive}  
  \caption{Наивное решение}
\label{fig:termination-naive}
\end{figure}

\paragraph{Алгоритм Дейкстры}
\label{sec:distr-term-dijkstra}

Для решения возникающей проблемы предлагается алгоритм, впервые описанный
Дейкстрой~\cite{DistrTerm}. Идея его заключается в том, что каждому узлу приписывается
цвет (в оригинале~--- белый и черный, здесь мы будем называть их красный и синий). Цвет
также приписывается аккумулятору, передаваемому между узлами.

Цвет определяется следующими правилами:
\begin{enumerate}
\item когда узел принимает новое сообщение и начинает его обработку~--- он становится красным;
\item когда узел отправляет аккумулятор следующему узлу~--- он становится синим;
\item первый узел отправляет синий аккумулятор;
\item когда красный узел отправляет аккумулятор, последний также становится красным.
\end{enumerate}

Распределенное завершение обнаруживается первым узлом, если принятый от последнего узла
аккумулятор~--- синий. Это означает, что ни один узел не занимался обработкой на
протяжении участка времени от предпоследней до последней отправки аккумулятора и,
следовательно, во время последнего <<витка>> аккумулятора ни один узел не посылал новых
сообщений. Наглядно работа этого алгоритма для того же сценария продемонстрирована на
рис.~\ref{fig:termination-dijkstra}.

\begin{figure}[htb]
  \centering
  \includegraphics[width=1\textwidth]{../graphics/termination-dijkstra}  
  \caption{Алгоритм Дейкстры}
\label{fig:termination-dijkstra}
\end{figure}

\section{Журналирование и отладка}
\label{sec:mpi-logging}

Для отладки параллельной программы необходима возможность отладочного журналирования.
Обычно на кластере у всех узлов есть общий доступ к дисковому пространству через сетевую
файловую систему (NFS или GPFS), поэтому можно выводить отладочные сообщения в файл на
диске.

Если каждый узел будет осуществлять вывод в журнал самостоятельно, высока вероятность
<<перекрытия>> строк от различных узлов. Можно использовать средства MPI-IO~\cite{MPI},
предоставляющие средства синхронизации доступа к файловой системе, однако, поскольку при
отладке высокая производительность не требуется, используется другое, более простой,
способ: при запуске в отладочной конфигурации один из узлов используется как служба
журналирования. Остальные узлы вместо непосредственного вывода отладочных сообщений в файл
посылают их в виде сообщений службе журналирования, который находится в постоянном цикле
приема таких сообщений и выводит их в журнал.

\section{Формат сообщений}
\label{sec:message-format}

Каждое сообщение в MPI, помимо данных, содержит тег (MPI tag), который может
использоваться, как идентификатор типа сообщения. В частности, возможен прием лишь
сообщений с конкретным тегом, однако данная функция используется лишь службой
журналирования (который не принимает никаких сообщений, кроме отладочного вывода).

Всего используется 4 типа сообщений, каждое имееющее свой тег.
\begin{itemize}
\item Сообщение с новым состоянием. Содержит состояние в виде массива байт
  (\Code{MPI_CHAR}). Представление самого состояния описано далее в
  разделе~\ref{sec:state-represent}.
\item Сообщение с счетчиком-аккумулятором (для обнаружения завершения,
  см.~раздел~\ref{sec:distributed-termination}). Содержит 2 элемента \Code{MPI_INT}:
  счетчик и его цвет (0~--- синий, 1~--- красный).
\item Сообщение с объявлением о завершении. Рассылается всем остальным узлам, когда
  какой-то узел обнаруживает завершение (это может быть первый узел, если завершение
  обнаружено алгоритмом Дейкстры, или любой другой, если завершение происходит из-за
  нахождения контпримера $\pi_e$). Состоит из одного элемента \Code{MPI_INT}~--- номера
  узла, обнаружившего завершение.
\item Сообщение с отладочным выводом. Должно посылаться лишь узлу со службой
  журналирования и обрабатывается лишь им. Содержит строку в виде символьного массива
  (\Code{MPI_CHAR}).
\end{itemize}

%%% Local Variables: 
%%% mode: latex
%%% TeX-master: "thesis"
%%% End: 
