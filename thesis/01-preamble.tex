\sloppy 

\EqInChapter
\TableInChapter
\PicInChapter

\usepackage[
bookmarks=true, colorlinks=true, unicode=true,
urlcolor=black,linkcolor=black, anchorcolor=black,
citecolor=black, menucolor=black, filecolor=bla,
]{hyperref}

\usepackage{cmap}       % теперь из pdf можно копипастить русский текст
\usepackage{underscore} % Ура! Теперь можно писать подчёркивание.
\usepackage{graphicx}	% Пакет для включения рисунков

\usefont{T2A}{ftm}{m}{} % Ужирнение начертания шрифта --- после чего
                        % выглдяит таймсоподобно и удобнее для чтения
                        % в плохих условиях.

\usepackage{pgf}
\usepackage{tikz}
\usetikzlibrary{arrows,automata,shapes}

% Значения по умолчанию для listings
\lstset{
  breakatwhitespace=true,% разрыв строк только на whitespacce
  breaklines=true,       % переносить длинные строки
  captionpos=b,          % подписи снизу
  extendedchars=\true,   % допуск utf
  inputencoding=utf8,
  numbers=left,          % нумерация слeва
  showspaces=false,      % показывать пробелы подчеркиваниями -- идиотизм 70-х годов
  showstringspaces=false,
  showtabs=false,        % и табы тоже
  stepnumber=1,
  tabsize=4              % кому нужны табы по 8 символов?..
}


% Стиль для псесдовода: строчки обычно короткие, поэтому размер шрифта побольше
\lstdefinestyle{pseudocode}{
  basicstyle=\small,
  frame=none,
  keywordstyle=\color{black}\bfseries\underbar,
  language=Pseudocode,
  numberstyle=\footnotesize
}

% Стиль для обычного кода: маленький шрифт
\lstdefinestyle{realcode}{
  basicstyle=\scriptsize,
  frame=none,
  numberstyle=\footnotesize
}

% Стиль для коротких кусков обычного кода: средний шрифт
\lstdefinestyle{simplecode}{
  basicstyle=\footnotesize,
  frame=none,
  numberstyle=\footnotesize
}

% Определим свой язык для написания псевдокодов на основе Python
\lstdefinelanguage[]{Pseudocode}[]{Python}{
  morekeywords={each,empty,wait,do},% ключевые слова добавлять сюда
  morecomment=[s]{\{}{\}},% комменты {а-ля Pascal} смотрятся нагляднее
  literate=% а сюда добавлять операторы, которые хотите отображать как мат. символы
    {->}{\ensuremath{$\rightarrow$}~}2%
    {<-}{\ensuremath{$\leftarrow$}~}2%
    {:=}{\ensuremath{$\leftarrow$}~}2%
}[keywords,comments]

%% \textsubscript : holy shit

\makeatletter
\newcommand\textsubscript[1]{\@textsubscript{\selectfont#1}}
\def\@textsubscript#1{{\m@th\ensuremath{_{\mbox{\fontsize\sf@size\z@#1}}}}}
\newcommand\textbothscript[2]{%
  \@textbothscript{\selectfont#1}{\selectfont#2}}
\def\@textbothscript#1#2{%
  {\m@th\ensuremath{%
    ^{\mbox{\fontsize\sf@size\z@#1}}%
    _{\mbox{\fontsize\sf@size\z@#2}}}}}
\def\@super{^}\def\@sub{_}

\catcode`^\active\catcode`_\active
\def\@super@sub#1_#2{\textbothscript{#1}{#2}}
\def\@sub@super#1^#2{\textbothscript{#2}{#1}}
\def\@@super#1{\@ifnextchar_{\@super@sub{#1}}{\textsuperscript{#1}}}
\def\@@sub#1{\@ifnextchar^{\@sub@super{#1}}{\textsubscript{#1}}}
\def^{\let\@next\relax\ifmmode\@super\else\let\@next\@@super\fi\@next}
\def_{\let\@next\relax\ifmmode\@sub\else\let\@next\@@sub\fi\@next}
\makeatother

% can't use textcomp

\newcommand\regsign{\textsuperscript{\textcircled{\scriptsize{R}}}}
\newcommand\copysign{\textsuperscript{\copyright}}
\newcommand\tmsign{\textsuperscript{\scriptsize{TM}}}

%%% Local Variables: 
%%% mode: latex
%%% TeX-master: "thesis"
%%% End: 
