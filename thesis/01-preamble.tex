\sloppy 

\EqInChapter
\TableInChapter
\PicInChapter

\usepackage[
bookmarks=true, colorlinks=true, unicode=true,
urlcolor=black,linkcolor=black, anchorcolor=black,
citecolor=black, menucolor=black, filecolor=bla,
]{hyperref}

\usepackage{cmap}       % теперь из pdf можно копипастить русский текст
\usepackage{underscore} % Ура! Теперь можно писать подчёркивание.
\usepackage{graphicx}	% Пакет для включения рисунков

\usefont{T2A}{ftm}{m}{} % Ужирнение начертания шрифта --- после чего
                        % выглдяит таймсоподобно и удобнее для чтения
                        % в плохих условиях.

\usepackage{pgf}
\usepackage{tikz}
\usetikzlibrary{arrows,automata}

\usepackage{listings}
\usepackage{listingsutf8}

\lstset{
  breakatwhitespace=true,
  breaklines=true,
  captionpos=b,
  extendedchars=\true,
  inputencoding=utf8,
  numbers=left,
  showspaces=false,
  showstringspaces=false,
  showtabs=false,
  stepnumber=1,
  tabsize=4
}

\lstdefinestyle{pseudocode}{
  basicstyle=\small,
  frame=none,
  keywordstyle=\color{black}\bfseries\underbar,
  language=Pseudocode,
  numberstyle=\footnotesize
}

\lstdefinestyle{realcode}{
  basicstyle=\scriptsize,
  frame=none,
  keywordstyle=\color{black}\bfseries,
  numberstyle=\footnotesize
}

\lstdefinestyle{simplecode}{
  basicstyle=\footnotesize,
  frame=none,
  keywordstyle=\color{black}\bfseries,
  numberstyle=\footnotesize
}

\lstdefinelanguage[]{Pseudocode}[]{Python}{
  morekeywords={each,empty},
  morecomment=[s]{\{}{\}},
  literate=
    {->}{\ensuremath{$\rightarrow$}~}2%
    {<-}{\ensuremath{$\leftarrow$}~}2%
}[keywords,comments]

\renewcommand*\thelstnumber{\oldstylenums{\the\value{lstnumber}}}
\renewcommand\lstlistingname{\cyr\CYRL\cyri\cyrs\cyrt\cyri\cyrn\cyrg}
\renewcommand\lstlistlistingname{\cyr\CYRL\cyri\cyrs\cyrt\cyri\cyrn\cyrg\cyri}

%%% Local Variables: 
%%% mode: latex
%%% TeX-master: "thesis"
%%% End: 
