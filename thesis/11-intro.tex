\Introduction

При разработке параллельных алгоритмов и протоколов взаимодействия часто возникает
необходимость их проверки на соответствие определённым спецификациям -- обычно это
отсутствие взаимоблокировок и каких-то других критических ситуаций.

Функциональное тестирование системы позволяет выявить не все ошибки, а лишь наиболее часто
встречающиеся. В то же время, определённые классы ошибок требуется полностью
исключить. Например, если новое расширение сетевого протокола с улучшенным контролем
трафика приводит к потере данных только в $10^{-3}\%$ случаев, это может остаться не
выявленным при тестировании, однако при использовании на 100 млн. машин приведёт к потерям
на 1000 из них, что уже довольно много. Для их выявления применяется \Term{проверка модели}
(model checking) -- автоматическая формальная верификация модели данной системы,
позволяющая проверить модель на соответствие спецификации и, в случае несоответствия,
предоставить контрпример.

Поскольку размер пространства состояний модели растет экспоненциально, даже для среднего
размера моделей оно насчитывает до $10^9 - 10^{10}$ состояний, требующих десятки гигабайт
памяти для хранения множества пройденных состояний. Это является главным препятствием для
полного перебора всех состояний модели. Применяются несколько подходов к сокращению объема
требуемой памяти: первичное упрощение модели, метод последовательных приближений,
различные оптимизации хранения состояний -- сжатие состояний, битовое хэширование и т.д. 

В данной работе предлагается другой подход -- параллельное выполнение на нескольких узлах
вычислительной сети с распределением хранимых состояний между ними.

%%% Local Variables: 
%%% mode: latex
%%% TeX-master: "main"
%%% End: 
