\Introduction

При разработке параллельных алгоритмов и протоколов взаимодействия возникает задача их
проверки на соответствие определённым спецификациям, таким, как отсутствие
взаимоблокировок и других критических ситуаций.

Тестирование ПО позволяет выявить не все ошибки, а лишь наиболее часто встречающиеся. В то
же время, определённые классы ошибок требуется полностью исключить. Например, если новое
расширение сетевого протокола с улучшенным контролем трафика приводит к потере данных
только в $10^{-3}\%$ случаев, это может остаться не выявленным при тестировании, однако
при использовании на 100 млн. машин приведёт к потерям на 1000 из них, что уже довольно
много. Для их выявления применяется \Term{проверка свойств модели} (model checking) --
автоматическая формальная верификация модели данной системы, позволяющая проверить модель
на соответствие спецификации и, в случае несоответствия, предоставить
контрпример.~\cite{Clarke}

Наиболее известным средством проверки свойств модели является ПО Spin, разработанное в
NASA JPL Laboratory for Reliable Software~\cite{SPIN}. ПО Spin осуществляет полный перебор
пространства состояний модели, проверяя на нем набор заданных утверждений и, в случае
нарушения, выдает контрпример~--- последовательность состояний, приведшую к нарушению
одного из утверждений.

Поскольку размер пространства состояний модели растет экспоненциально, даже для среднего
размера моделей оно насчитывает до $10^9 - 10^{10}$ состояний, требующих десятки гигабайт
памяти для хранения множества пройденных состояний.~\cite{SpinRoot} Это является главным
препятствием для полного перебора всех состояний модели. Применяются несколько подходов к
сокращению объема требуемой памяти: первичное упрощение модели, метод последовательных
приближений, различные оптимизации хранения состояний -- сжатие состояний, битовое
хэширование и т.д.~\cite{Katoen}

В данной работе предлагается другой подход~--- параллельное выполнение на нескольких узлах
вычислительной сети с распределением хранимых состояний между ними.

Целью работы является улучшение временных характеристик за счет параллельной генерации
состояний и увеличения объема памяти, доступной для хранения пространства состояний.

Для достижения поставленной цели необходимо:
\begin{enumerate}
\item проанализировать проблемы проверки моделей и существующие подходы к их решению;
\item разработать и реализовать алгоритм параллельной генерации состояний;
\item спроектировать схему автоматического распределения состояний между узлами;
\item провести эксперименты и проанализировать их результаты.
\end{enumerate}

%%% Local Variables: 
%%% mode: latex
%%% TeX-master: "thesis"
%%% End: 
