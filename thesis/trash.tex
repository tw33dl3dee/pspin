
% \section{Тестирование основных модулей программного комплекса}
% В данном разделе приводится описание результатов модульного тестирования программного комплекса. Тестируется 
% вспомогательный функционал, не участвующий непосредственно в процессе верификации.

% \textbf{Загрузка конфигурационных файлов}

% Классы эквивалентности для действия "Загрузка конфигурационных файлов" приведены в табл. 
% \ref{tab:equclass_conf_load}.

% \begin{table}[!ht]
%   	\caption{Классы эквивалентности для действия "Загрузка конфигурационных данных"}
% 	\begin{tabular}{|p{8cm}|p{8cm}|}
% 		\hline
% 			Допустимые классы эквивалентности & Недопустимые классы эквивалентности \\
% 		\hline
% 			Все существующие и синтаксически корректные YAML-файлы & Некорректные YAML-файлы; несуществующие файлы \\
% 		\hline
% 	\end{tabular}
% 	\label{tab:equclass_conf_load}
% \end{table}

% Описания тестов для данного действия приведены в табл. \ref{tab:testcase_conf_load}.
% \begin{table}[!ht]
%   	\caption{Тестовые случаи для действия "Загрузка конфигурационных данных"}
% 		\begin{tabular}{|p{4cm}|p{4cm}|p{3cm}|p{3cm}|}
% 			\hline
% 			Описание & Входные данные & Ожидаемый результат & Полученный результат \\
% 			\hline
% 			Несуществующий файл& "" & Сообщение "Файл не существует" & Сообщение "Файл не существует" \\
% 			\hline
% 			Пустой файл& "" & Корректный файл & Корректный файл\\
% 			\hline
% 			Файл с некорректной табуляцией& "vars:$\backslash$n$\backslash$ta:$\backslash$n $\backslash$ttype: 'double'" & Сообщение "Ошибка в строке 3" & Сообщение "Ошибка в строке 3"\\
% 			\hline
% 			Корректный заполненный файл& "vars:$\backslash$n$\backslash$ta:$\backslash$n $\backslash$t$\backslash$ttype: 'double' & Корректный файл & Корректный файл\\
% 			\hline
% 	\end{tabular}
% 	\label{tab:testcase_conf_load}
% \end{table}

% \textbf{Проверка структуры модели}

% Классы эквивалентности для действия "Проверка структуры модели" приведены в табл. 
% \ref{tab:equclass_conf}.

% \begin{table}[!ht]
%   	\caption{Классы эквивалентности для действия "Проверка структуры модели"}
% 	\begin{tabular}{|p{8cm}|p{8cm}|}
% 		\hline
% 		Допустимые классы эквивалентности & Недопустимые классы эквивалентности \\
% 		\hline
% 		Корректно описанные модели (в соответствии с разделом 3.2) & Некорректно описанные модели \\
% 		\hline
% 	\end{tabular}
% 	\label{tab:equclass_conf}
% \end{table}

% Описания тестов для данного действия приведены в табл. \ref{tab:testcase_conf}.

% \begin{table}[!ht]
%   	\caption{Тестовые случаи для действия "Проверка структуры модели"}
% 	\begin{tabular}{|p{5cm}|p{5cm}|p{5cm}|}
% 		\hline
% 		Описание & Ожидаемый результат & Полученный результат \\
% 		\hline
% 		Модель без секции с описанием состояний & 
% 		Сообщение "Наличие описания состояний обязательно" & 
% 		Сообщение "Наличие описания состояний обязательно" \\
% 		\hline
% 		Модель без секции с описанием входных данных &
% 		Сообщение "Наличие описания входов обязательно" &  
% 		Сообщение "Наличие описания входов обязательно"\\
% 		\hline
% 		Вероятности переходов для одного из состояний имеет сумму, б\textit{о}льшую 1 & 
% 		Сообщение "Сумма указанных вероятностей не должна превышать 1" &
% 		Сообщение "Сумма указанных вероятностей не должна превышать 1"\\
% 		\hline
% 		Корректная модель & 
% 		Проверка пройдена &
% 		Проверка пройдена \\
% 		\hline		
% 	\end{tabular}
% 	\label{tab:testcase_conf}
% \end{table}

% \textbf{Проверка параметров командной строки}

% Классы эквивалентности для действия "Проверка параметров командной строки" приведены в табл. 
% \ref{tab:equclass_console}.

% \begin{table}[!ht]
%   	\caption{Классы эквивалентности для действия "Проверка структуры модели"}
% 	\begin{tabular}{|p{8cm}|p{8cm}|}
% 		\hline
% 		Допустимые классы эквивалентности & Недопустимые классы эквивалентности \\
% 		\hline
% 		Корректные параметры (в соответствии с разделом 3.6) & Некорректные параметры \\
% 		\hline
% 	\end{tabular}
% 	\label{tab:equclass_console}
% \end{table}

% Описания тестов для данного действия приведены в табл. \ref{tab:testcase_console}.

% \begin{table}[!ht]
%   	\caption{Тестовые случаи для действия "Проверка структуры модели"}
% 	\begin{tabular}{|p{4cm}|p{4cm}|p{3cm}|p{3cm}|}
% 		\hline
% 		Описание & Входные данные & Ожидаемый результат & Полученный результат \\
% 		\hline
% 		Недопустимый параметр &
% 		"-u 170" &
% 		Сообщение "Недопустимый параметр: u" & 
% 		Сообщение "Недопустимый параметр: u" \\
% 		\hline
% 		Недопустимое значени &
% 		"-cway z" &
% 		Сообщение "Параметр cway имеет недопустимое значени: z" &  
% 		Сообщение "Параметр cway имеет недопустимое значени: z"\\
% 		\hline
% 		Вызов без параметров & 
% 		"" &
% 		Корректное поведение (берутся значения по умолчанию) &
% 		Корректное поведение (берутся значения по умолчанию)\\
% 		\hline
% 		Вызов с одним параметром & 
% 		"-r" &
% 		При генерации наборов используется только случайный метод&
% 		При генерации наборов используется только случайный метод \\
% 		\hline		
% 	\end{tabular}
% 	\label{tab:testcase_console}
% \end{table}

%% USE CASE
% В процессе анализа предметной области и поставленных задач выявлены следующие прецеденты, 
% показанные на рис. \ref{use_case}:

% \begin{itemize}
% \item подготовка исходной модели;
% \item управления параметрами верификации;
% \item проведение верификации;
% \item выдача результатов.
% \end{itemize}

% \subsection{Подготовка исходной модели}
% Прецедент служит для создания и модификации моделей: добавление, изменение и удаление информации в текстовом 
% режиме.

% Предусловия: нет.

% \textit{Основной сценарий.} 

% \begin{enumerate}
% \item Пользователь формирует конфигурационный файл в соответствии с требованиями к его составу.
% \item Пользователь указывает программе путь к конфигурационному файлу.
% \item Программа загружает и обрабатывает файл.
% \item Программа выдает сообщение об успешной обработке файла.
% \end{enumerate}

% \textit{Альтернативный сценарий 1.}
% \begin{enumerate}
% \item Пользователь формирует конфигурационный файл в соответствии с требованиями к его составу.
% \item Пользователь указывает программе путь к конфигурационному файлу.
% \item Программа не обнаруживает файл по указанному пути, так как он не верен.
% \item Программа выдает сообщение об ошибке "Некорректный путь к файлу".
% \end{enumerate}

% \textit{Альтернативный сценарий 2.}
% \begin{enumerate}
% \item Пользователь формирует конфигурационный файл в соответствии с требованиями к его составу.
% \item Пользователь указывает программе путь к конфигурационному файлу.
% \item Программа загружает и обрабатывает файл.
% \item Программа обнаруживает синтаксическую ошибку в структуре файла.
% \item Программа выдает сообщение об ошибке "Синтаксическая ошибка" с указанием строки, содержащей ошибку.
% \end{enumerate}

% \textit{Альтернативный сценарий 3.}
% \begin{enumerate}
% \item Пользователь формирует конфигурационный файл в соответствии с требованиями к его составу.
% \item Пользователь указывает программе путь к конфигурационному файлу.
% \item Программа загружает и обрабатывает файл.
% \item Программа обнаруживает семантическую ошибку (не описаны обязательные блоки) в структуре файла.
% \item Программа выдает сообщение об ошибке "Не описан блок" с указанием требуемого блока.
% \end{enumerate}

%%% Local Variables: 
%%% mode: latex
%%% TeX-master: "thesis"
%%% End: 
