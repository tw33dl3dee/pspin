\Introduction

При разработке параллельных алгоритмов и протоколов взаимодействия
часто возникает необходимость их верификации на соответствие
определённым условиям целостности. Например, параллельного алгоритма
не должен иметь взаимоблокировок, а в протоколе взаимодействия все
участвующие стороны должны всегда достигать конечного состояния.

Функциональное тестирование системы позволяет выявить не все ошибки, а
лишь наиболее часто встречающие. В то же время, определённые классы
ошибок требуется полностью исключить. Например, если новое расширение
сетевого протокола с улучшенным контролем трафика приводит к потере
данных только в $10^{-3}\%$ случаев, это может остаться не выявленным при
тестировании, однако при использовании на 100 млн. машин приведёт к
потерям на 1000 из них, что уже довольно много. Для их выявления и
подходит верификация самой модели, которая путем перебора всех
состояний системы проверит, возможен ли такой сценарий
функционирования системы, при котором она приходит в недопустимое
состояние.

Основным программным средством для подобной проверки конечных моделей
является \Term{SPIN}. Для описания модели в нем используется язык описания
Promela (\Term{PROtocol Meta Language}). 

При верификации модели SPIN выполняет исчерпывающий поиск в глубину по
пространству состояний и, при достижении недопустимого состояния,
сохраняет сценарий, приведший к приходу в это состояние.

Поскольку размер пространства состояний модели растет экспоненциально,
даже для среднего размера моделей оно насчитывает до $10^9 - 10^{10}$
состояний, требующих десятки гигабайт памяти для хранения множества
пройденных состояний. Это делает проверку больших моделей
нереализуемой практически. Для сокращения объема требуемой памяти
используются различные оптимизации — сжатие хранимых состояний,
битовое хэширование пространства состояний и т.д. В
данной работе предлагается другой подход — параллельное выполнение на
нескольких узлах вычислительной сети с распределением хранимых
состояний между ними.

%%% Local Variables: 
%%% mode: latex
%%% TeX-master: "report"
%%% End: 
