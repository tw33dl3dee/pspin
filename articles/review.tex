\documentclass[a4paper,12pt,notitlepage]{article}

\usepackage{ucs}
\usepackage[utf8x]{inputenc}
\usepackage[T1]{fontenc}
\usepackage[english,russian]{babel}

\usepackage{textcomp}
\usepackage{indentfirst}
\usepackage{verbatim}
\usepackage{setspace}
\usepackage[left=25mm,right=20mm,top=10mm,bottom=25mm,headsep=0pt]{geometry}

\sloppy

%\onehalfspacing

\title{Рецензия}

\date{\normalsize{<<Параллельная генерация состояний конечной модели для проверки ее
    свойств>>}}

\author{\small{на дипломный проект студента факультета <<Информатика и системы
    управления>>} \\ \small{группы ИУ~7--122 Короткова~И.\,А. на тему}}

\begin{document}

\maketitle

\thispagestyle{empty}

Формальная верификация (проверка модели) крайне важна при разработке алгоритмов и
протоколов в областях, где необходимо гарантировать высокую степень надежности:
телекоммуникационных системах, сетевых протоколах и управляющих системах, работащих в
реальном времени. Высокая ресурсоемкость этого подхода накладывает большие ограничения на
возможность его применения, поэтому данная работа является актуальной.

В дипломной работе рассматриваются существующие подходы к проверке моделей и способы
решения проблемы экспоненциального роста числа состояния. Предлагается свое решение~---
параллельная генерация состояний с распределенным их хранением на всех узлах кластера.

Проектируется алгоритм для параллельной генерации и хранения состояний. Особое внимание
уделяется проблеме выбора распределения состояний между узлами, которое бы позволяло
добиться не только равномерной загрузки узлов, но и уменьшить число передаваемых сообщений
между ними.

Спроектированный алгоритм параллельной генерации реализован на должном профессиональном
уровне. Благодаря автоматической генерации кода и тщательной оптимизации получаемый код по
скорости работы находится на одном уровне с верификатором SPIN, широко используемым в
данной области. Поддерживается верификация моделей, написанных на языке PROMELA, что
позволяет использовать созданное средство для проверки уже существующих моделей без
существенной модификации.

Проведенные эксперименты подтверждают, что идея параллельной генерации является подходящим
решением проблемы экспоненциального роста пространства состояний и расширяет возможности
проверки моделей в практических задачах.

К недостаткам работы стоит отнести отсутствие поддержки формул временной логики LTL, часто
используемых в проверке моделей, и выдачу в качестве контрпримера лишь отдельного
состояния вместо полной трассы.

Приведенные недостатки объясняются ограниченным объемом работы и не снижают ее
практической и научной ценности. Дипломная работа соответствует квалификационным
требованиям и заслуживает отличной оценки, а Коротков~И.\,А.~--- присвоения степени
магистра по направлению 23010068~<<Магистр и технологии>>.

\begin{table}[!b]
  \begin{tabular*}{1\textwidth}{@{\extracolsep{\fill}}p{0.15\textwidth}p{0.3\textwidth}r}
    \multicolumn{3}{l}{Начальник отдела системного программирования НИИСИ~РАН,~к.т.н.} \\ \\
    Годунов~А.\,Н. & 
    \underline{~~~~~~~~~~~~~~~~~~~~~~~~~} & 
    <<\underline{~~~~~}>>\,\underline{~~~~~~~~~~~~~~~~~~~~~~~~~}\,2010~г.
  \end{tabular*}
\end{table}

\end{document}

%%% Local Variables: 
%%% mode: latex
%%% TeX-master: t
%%% End: 
