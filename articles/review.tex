\documentclass[a4paper,12pt,notitlepage]{article}

\usepackage{ucs}
\usepackage[utf8x]{inputenc}
\usepackage[T1]{fontenc}
\usepackage[english,russian]{babel}

\usepackage{textcomp}
\usepackage{indentfirst}
\usepackage{verbatim}
\usepackage{setspace}
\usepackage[left=30mm,right=25mm,top=10mm,bottom=25mm,headsep=0pt]{geometry}

\sloppy

%\onehalfspacing

\title{Рецензия}

\date{\normalsize{<<Параллельная генерация состояний конечной модели для проверки ее
    свойств>>}}

\author{\small{на дипломную работу магистра Короткова Ивана Андреевича}}

\begin{document}

\maketitle

\thispagestyle{empty}

Проверка конечных моделей методом прямого перебора их состояний может применяться при
разработке алгоритмов и протоколов в областях, где необходимо гарантировать высокую
степень надежности: телекоммуникационных системах, сетевых протоколах и управляющих
системах, работащих в реальном времени. Высокая ресурсоемкость этого подхода накладывает
большие ограничения на возможность его применения, поэтому поставленная задача является
актуальной.

В представленной дипломной работе рассмотрены существующие подходы к проверке моделей и
способы решения проблемы экспоненциального роста числа состояния. Предложено свое
решение~--- параллельная генерация состояний с распределенным их хранением на всех узлах
кластера.

В работе спроектирован алгоритм для параллельной генерации и хранения состояний. Особое
внимание уделено проблеме выбора распределения состояний между узлами, которое бы
позволяло добиться не только равномерной загрузки узлов, но и уменьшить число передаваемых
сообщений между ними.

Спроектированный алгоритм параллельной генерации реализован на должном профессиональном
уровне. Благодаря автоматической генерации кода и тщательной оптимизации получаемый код по
скорости работы находится на одном уровне с верификатором Spin, широко используемым в
данной области. Поддерживается верификация моделей, написанных на языке Promela, что
позволяет использовать созданное средство для проверки уже существующих моделей без
существенной модификации.

Проведенные эксперименты показывают применимость разработанного метода для проверки
моделей с большим числом состояний, возникающих в практических задачах.

К недостаткам работы стоит отнести отсутствие поддержки условий корректности в виде формул
временной логики LTL, часто используемых в проверке моделей, и выдачу в качестве
контрпримера лишь отдельного состояния вместо полной трассы.

Указанные недостатки не снижают практической и научной ценности работы. Дипломная работа
соответствует квалификационным требованиям и заслуживает отличной оценки, а
Коротков~И.\,А.~--- присвоения квалификации магистра техники и технологии по направлению
552800~<<Информатика и вычислительная техника>>.

\begin{table}[!b]
  \begin{tabular*}{1\textwidth}{@{\extracolsep{\fill}}p{0.15\textwidth}p{0.3\textwidth}r}
    \multicolumn{3}{l}{Зав.~отделом системного программирования НИИСИ~РАН,~к.ф.-м.н.} \\ \\
    Годунов~А.\,Н. & 
    \underline{~~~~~~~~~~~~~~~~~~~~~~~~~} & 
    <<\underline{~~~~~}>>\,\underline{~~~~~~~~~~~~~~~~~~~~~~~~~}\,2010~г.
  \end{tabular*}
\end{table}

\end{document}

%%% Local Variables: 
%%% mode: latex
%%% TeX-master: t
%%% End: 
