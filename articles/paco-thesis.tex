\documentclass[12pt,a4paper,notitlepage]{article}

\usepackage{ucs}
\usepackage[utf8x]{inputenc}
\usepackage[T1]{fontenc}
\usepackage[english,russian]{babel}

\usepackage{textcomp}
\usepackage{indentfirst}
\usepackage{verbatim}

\usepackage[left=20mm,right=20mm,top=20mm,bottom=20mm]{geometry}

%\usepackage{cyrtimes}

% \usepackage{listings}
% \usepackage{listingsutf}

% % Значения по умолчанию для listings
% \lstset{
%   basicstyle=\scriptsize,
%   frame=none,
%   numberstyle=\footnotesize,
%   breakatwhitespace=true,% разрыв строк только на whitespacce
%   breaklines=true,       % переносить длинные строки
%   captionpos=b,          % подписи снизу
%   extendedchars=\true,   % допуск utf
%   inputencoding=utf8,
%   numbers=left,          % нумерация слeва
%   showspaces=false,      % показывать пробелы подчеркиваниями -- идиотизм 70-х годов
%   showstringspaces=false,
%   showtabs=false,        % и табы тоже
%   stepnumber=1,
%   tabsize=4              % кому нужны табы по 8 символов?..
% }

\sloppy
\hyphenpenalty=9000

\begin{document}

\begin{center}\large \bfseries Разработка параллельного алгоритма генерации состояний при
  проверке моделей для систем с неразделяемой памятью \end{center}

При создании параллельных алгоритмов или сетевых протоколов возникает задача их
верификации~--- поиска возможных взаимоблокировок, непредвиденных сценариев работы и
нарушения условий работоспособности. Одним из подходов к решению этой задачи является
автоматическая формальная верификация полным перебором состояний системы, называемая
\emph{проверкой модели}~\cite{Clarke}.

Одним из наиболее распространенных способов описания таких моделей является язык
Promela. Модель в последнем описывается в виде набора процессов, имеющих доступ к общим
переменным и использующим очереди для передачи сообщений. Проверяемые утверждения
выражаются в виде простых утверждений о значениях переменных или при помощи составления из
таких утверждений LTL-формул~\cite{Clarke}.

Поскольку при проверке модели обходится весь граф состояний, который в общем случае имеет
циклы, необходимо хранить в памяти множество посещенных состояний.

\textbf{Актуальность}. Для моделей, описывающих взаимодействие процессов или участников
сетевого протокола, наблюдается комбинаторный рост числа состояний. Модели среднего
размера, например, модель RIP-протокола для системы из 4 маршрутизаторов, насчитывает
порядка $10^{8}$ состояний, что делает невозможным хранение их в памяти одной машины, а
использование swap-памяти увеличивает время проверки на несколько порядков.

Наиболее распространенным средством проверки моделей является Spin~\cite{SPIN}, строящий
множество состояний последовательным образом на одной машине, и, следовательно,
неприменимый для проверки моделей такого размера. Имеется один экспериментальный аналог,
реализующий параллельную генерацию~--- DiVinE, разработанный в университете
Брно~\cite{DLTL2}.

\textbf{Постановка задачи}. Разработка параллельного алгоритма генерации состояний для
проверки модели с распределенным хранением состояний в различных узлах кластера.

Для решения поставленной задачи было сделано следующее: 

\begin{itemize}
\item создан прототип верификатора моделей, описываемых на языке Promela, с поддержкой
  утверждений в виде предположений о значениях переменных (assert);
\item предложено несколько способов распределения состояний между узлами, проведено их
  экспериментальное сравнение.
\end{itemize}

Основной проблемой распределенной генерации множества состояний является выбор подходящей
функции распределения состояний между узлами~\cite{LS99}. Такая функция должна быть
однозначно вычислимой по самому состоянию и обеспечивать равномерное распределение
состояний между узлами.

Тривиальным решением является является использовать хэш-код самого состояния в качестве
индекса хранящего его узла. При этом решении достигается высокая равномерность
распределения (до 99.9\%). Однако, новое состояние с большой вероятностью будет
принадлежать не тому узлу, где оно было сгенерировано и число удаленных вызовов между
узлами будет приближаться к числу переходов.

При альтернативном подходе индексом узла является хэш-код от состояния первых $M$
процессов моделируемой системы, т.е. значений их локальных переменных и номеров текущих
инструкций. В этом случае удаленный вызов делается лишь в результате переходов,
затрагивающих состояния этих процессов. Значение $M$ выбирается с учетом числа
вычислительных узлов и процессов в моделируемой системе. Слишком маленькие значения
приводят к неравномерному распределению, при больших растет число удаленных вызовов.

\textbf{Основные результаты}. Реализован параллельный генератор состояний на основе алгоритма
параллельного поиска в ширину с использованием указанной функции распределения.

Эксперименты на кластере из 80 узлов, имеющих 4 Гбайт памяти каждый, показывают
пригодность созданного генератора для проверки моделей с числом состояний до $10^{11}$. В
качестве моделей использовалась задача об обедающих философов и алгоритм распределенного
выбора лидера с различным числом сторон.

Проведена экспериментальная проверка предложенного распределения состояний, из которой
следует, что при подходящем выборе $M$ выигрыш во времени генерации состояний составляет
до 80\% в сравнении с тривиальным решением при неравномерности распределения между узлами
меньше 30\%.

\textbf{Научная новизна работы} Полученные результаты позволяют использовать
автоматическую формальную верификацию для проверки моделей б\'{о}льшего размера, чем это
возможно при использовании существующих средств.

\bibliographystyle{../thesis/gost780u}
\bibliography{../thesis/thesis}

\newpage

\small
\begin{tabular}{ll}
  Название доклада    & Разработка параллельного алгоритма генерации состояний \\
                      & при проверке моделей для систем с неразделяемой памятью \\
  ФИО, ученая степень & Коротков Иван Андреевич \\
  Должность, организация & аспирант, МГТУ им. Н.\,Э.\,Баумана \\
  Адрес организации & 1005005, Москва, 2-ая Бауманская ул., д.~5 \\
  Телефон           & +7\,(916)\,213--40--98 \\
  Адрес эл.~почты   & \texttt{twee@tweedle-dee.org}
\end{tabular}

\end{document}

%%% Local Variables: 
%%% mode: latex
%%% TeX-master: t
%%% End: 
