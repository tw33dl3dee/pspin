\documentclass[a4paper,12pt,notitlepage]{article}

\usepackage{ucs}
\usepackage[utf8x]{inputenc}
\usepackage[T1]{fontenc}
\usepackage[english,russian]{babel}

\usepackage{textcomp}
\usepackage{indentfirst}
\usepackage{verbatim}
\usepackage{setspace}
\usepackage[left=20mm,right=20mm,top=20mm,bottom=20mm,headsep=0pt]{geometry}

\sloppy

\onehalfspacing

\title{Отзыв}

\date{\normalsize{<<Параллельная верификация формул логики ветвящегося времени на
    дискретных детерминированных моделях моделях>>}}

\author{\small{на реферат магистра Короткова Ивана Андреевича}}

\begin{document}

\maketitle

\thispagestyle{empty}

Реферат Короткова Ивана Андреевича <<Параллельная верификация формул логики ветвящегося
времени на дискретных детерминированных моделях моделях>> достаточно полный, включает в
себя формулировку целей и задач, описание проблемы и обоснование ее актуальности, основные
ожидаемые результаты и научную новизну. Приведен библиографический и краткий теоретический
обзор по выбранному направлению, из которых следует, что автор владеет темой в полном
объеме и им была проведена предварительная исследовательская работа.

Реферат содержит хороший обзор современного состояния дел в выбранной области работ и
может служить основой для выбора тематики научных исследований.

Предложенная в реферате тематика работы может быть положена в основу работы над
диссертацией на звание кандидата технических наук по специальности 05.13.17
<<Теоретические основы информатики>>.

%%% --- ты владеешь темой в полном объеме на современном уровне
%%% --- проведен хороший обзор совоременного состояния дел
%%% --- предложенная тематика работы может быть положена в основу работы налд диссертацией
%%% на звание кандитата тезхнических наук по специальности 05.13.17 или 05.13.11 (почитай
%%% паспорта и выбери, кажется 17) 

\end{document}

%%% Local Variables: 
%%% mode: latex
%%% TeX-master: t
%%% End: 
