\documentclass[a4paper,12pt,notitlepage]{article}

\usepackage{ucs}
\usepackage[utf8x]{inputenc}
\usepackage[T1]{fontenc}
\usepackage[english,russian]{babel}

\usepackage{textcomp}
\usepackage{indentfirst}
\usepackage{verbatim}
\usepackage{setspace}
\usepackage[left=20mm,right=20mm,top=20mm,bottom=20mm,headsep=0pt]{geometry}

\sloppy

\onehalfspacing

\title{Отзыв}

\date{\normalsize{<<Параллельная верификация формул логики ветвящегося времени на
    дискретных детерминированных моделях моделях>>}}

\author{\small{на научный реферат Короткова Ивана Андреевича}}

\begin{document}

\maketitle

\vspace{3cm}

\thispagestyle{empty}

Реферат Короткова Ивана Андреевича <<Параллельная верификация формул логики ветвящегося
времени на дискретных детерминированных моделях моделях>> содержит описание проблемы и
обоснование ее актуальности, формулировку целей и задач, которые могут быть решены в
рамках обучения в аспирантуре. Приведены основные ожидаемые результаты и научная новизна
предполагаемой работы.

Насколько можно судить по приведенному библиографическому и краткому теоретическому обзору
по выбранному направлению, автор владеет темой параллельной проверки на модели в полном
объеме и им была проведена предварительная аналитическая работа.

Реферат содержит подробный обзор современного состояния дел в выбранной области работ и
может служить основой для выбора тематики научных исследований.

Предложенная в реферате тематика работы может быть положена в основу работы над
диссертацией на соискание звания кандидата технических наук по специальности 05.13.17
<<Теоретические основы информатики>>.

% ЗАКЛЮЧЕНИЕ по научному реферату Ребрикова Александра Викторовича на тему «Неполная
% верификация сложных дискретных систем» Сформулированная в реферате постановка задачи и
% возможные варианты ее решения являются логическим продолжением темы магистерской работы
% РЕБРИКОВА Александра Викторовича.  Задача содержит актуальный компонент её инженерной
% реализации, которая позволяет существенно оптимизировать затраты на проведение
% верификации программного обеспечения. В работе затрагиваются проблемы в области
% символьного решения систем логических ограничений. Ребриков Александр занимается этой
% проблемой в течение продолжительного времени, и им уже достигнуты следующие результаты:
% в реферате приведен анализ предметной области, дана оценка наиболее распространенных
% программных продуктов, сформулирована гипотеза о возможном пути решения проблемы. В
% рамках своей магистерской работы им уже создан программный комплекс, который позволят
% автоматизировать процедуру верификации программного обеспечения. Для достижения
% сформулированной в реферате цели Ребрикову Александру необходимо решить задачи, которые
% могут составить основу будущей диссертации. Следует подчеркнуть интерес Ребрикова
% Александра к теме исследования. Проявленное во время обучения на дневном отделении
% трудолюбие, добропорядочность и склонность к исследовательской деятельности позволяют
% говорить об успешном выполнении им диссертационного исследования.

% ____Рудаков Игорь Владимирович, к.т.н., доцент ___ ________________ ф.и.о., степень,
% звание подпись 

\noindent
Заведующий кафедрой ИУ7, д.т.н \hspace{7cm}Б.Г. Трусов

%\begin{table}[!b]
%  \begin{tabular*}{1\textwidth}{@{\extracolsep{\fill}}p{0.6\textwidth}r}
%    \multicolumn{3}{l}{} \\ \\
%    Трусов Борис Георгиевич,~д.т.н.,~профессор & 
%    \underline{~~~~~~~~~~~~~~~~~~~~~~~~~~~~~~~~~~~}  \\ \\
%    & <<\underline{~~~~~}>>\,\underline{~~~~~~~~~~~~~~~~~~~~~~~}\,2010~г.
%  \end{tabular*}
%\end{table}

\end{document}

%%% Local Variables: 
%%% mode: latex
%%% TeX-master: t
%%% End: 
