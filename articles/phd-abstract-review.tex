\documentclass[a4paper,12pt,notitlepage]{article}

\usepackage{ucs}
\usepackage[utf8x]{inputenc}
\usepackage[T1]{fontenc}
\usepackage[english,russian]{babel}

\usepackage{textcomp}
\usepackage{indentfirst}
\usepackage{verbatim}
\usepackage{setspace}
\usepackage[left=20mm,right=20mm,top=20mm,bottom=20mm,headsep=0pt]{geometry}

\sloppy

\onehalfspacing

\title{Отзыв}

\date{\normalsize{<<Параллельная верификация формул логики ветвящегося времени на
    дискретных детерминированных моделях моделях>>}}

\author{\small{на реферат Короткова Ивана Андреевича}}

\begin{document}

\maketitle

\vspace{3cm}

\thispagestyle{empty}

Реферат Короткова Ивана Андреевича <<Параллельная верификация формул логики ветвящегося
времени на дискретных детерминированных моделях моделях>> содержит описание проблемы и
обоснование ее актуальности, формулировку целей и задач, которые могут быть решены в
рамках обучения в аспирантуре. Приведены основные ожидаемые результаты и научная новизна
предполагаемой работы.

Насколько можно судить по приведенному библиографическому и краткому теоретическому обзору
по выбранному направлению, автор владеет темой параллельной проверки на модели в полном
объеме и им была проведена предварительная аналитическая работа.

Реферат содержит подробный обзор современного состояния дел в выбранной области работ и
может служить основой для выбора тематики научных исследований.

Предложенная в реферате тематика работы может быть положена в основу работы над
диссертацией на соискание звания кандидата технических наук по специальности 05.13.17
<<Теоретические основы информатики>>.

\begin{table}[!b]
  \begin{tabular*}{1\textwidth}{@{\extracolsep{\fill}}p{0.4\textwidth}p{0.2\textwidth}r}
%    \multicolumn{3}{l}{} \\ \\
    Профессор,~д.т.н. Трусов~Б.\,Г. & 
    \underline{~~~~~~~~~~~~~~~~~~~~~~~} & 
    <<\underline{~~~~~}>>\,\underline{~~~~~~~~~~~~~~~~~~~~~~~}\,2010~г.
  \end{tabular*}
\end{table}

\end{document}

%%% Local Variables: 
%%% mode: latex
%%% TeX-master: t
%%% End: 
