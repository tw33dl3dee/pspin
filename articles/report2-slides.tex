\documentclass[12pt]{article}

\usepackage[utf8]{inputenc}
\usepackage[english, russian]{babel}

\usepackage{slides}

% ШРИФТЫ
% Нужны рубленные шрифты -- раскомментируйте стоку ниже. 
% Нужны шрифты с засечками --- закомментируйте эту строку. 
\renewcommand{\familydefault}{\sfdefault} % Переключает на рубленный шрифт.
% Шрифты Times и Arial, если стоит пакет cyrtimes. 
% Если он не стоит, результат будет плохой!
%\usepackage{cyrtimespatched}
% Если нет cyrtimes, то попробуйте включить полужирный шрифт:
% \renewcommand{\seriesdefault}{b} % для шрифта с засечками, это предпочтительно
% \renewcommand{\seriesdefault}{sbc} % для рубленного шрифта

\usepackage{graphicx}

\def\Student{Коротков Иван Андреевич}
\def\Advisor{Крищенко Всеволод Александрович}
\def\Title{Параллельное построение пространства состояний конечной модели}
\def\SubTitle{Квалификационная работа}

% Верхний заголовок: пустой
% Нижний заголовок по-умолчанию:
% \lfoot{\Title} % слева
% \cfoot{} % цент пуст
% \rfoot{\thepage} % справа

% \renewcommand{\baselinestretch}{1.5}
 \linespread{1.2}

\begin{document}

\TitleSlide

\section{Ошибки в ПО}
\label{sec:sw-errors}

\begin{tabular}[ht]{|p{0.3\textwidth}|p{0.3\textwidth}|p{0.3\textwidth}|}
  \hline 
  Вероятность возникновения & Критические & Некритичные \\
  \hline
  высокая & Тестирование, верификация & Тестирование \\
  \hline
  низкая & Верификация & \\
  \hline
\end{tabular}

\section{Проблемы верификации}
\label{sec:verif-troubles}

\begin{itemize}
\item Размер пространства состояний растет экспоненциально

\item Модели среднего размера требуют более 10 Гб оперативной памяти для хранения пространства состояний

\item На верификацию средней модели уходят недели
\end{itemize}

\section{Уменьшение требований к оперативной памяти}
\label{sec:ram-lightening}

Генерация состояний:

\begin{itemize}
\item Последовательная
  \begin{itemize}
  \item Битовое хэширование
  \item Сжатие состояний
  \item Сокращение частных порядков
  \end{itemize}
\item Параллельная
\end{itemize}

\section{Параллельная генерация}
\label{sec:par-gen}

\begin{itemize}
\item Распределенное хранилище состояний
  \begin{itemize}
  \item Генерация состояний только на одном узле
  \item Неэффективно из-за неиспользования вычислительных ресурсов и большого количества
    удаленных вызовов
  \item Позволяет выполнять обход в глубину и поиск циклов
  \end{itemize}

\item Распределенная генерация состояний
  \begin{itemize}
  \item Каждый узел является хранилищем и генератором
  \item Используются вычислительные ресурсы всех узлов
  \item Необходимо использовать поиск в ширину: поиск циклов затруднен
  \end{itemize}

\end{itemize}

\section{Распределение состояний между узлами}
\label{sec:state-partitioning}

\includegraphics[width=0.9\textwidth]{../graphics/state-partition}

\section{Сравнение распределений}
\label{sec:partition-compare}

\begin{tabular}[ht]{|r|l|l|p{0.25\textwidth}|p{0.25\textwidth}|}
  \hline 
  P & Состояния & Переходы & Удаленные вызовы & Загруженность узлов    \\ \hline
  5 & 28351     & 42658    & 31356  (73\%)/ 7047  (16\%) & 0.99 / 0.66 \\ \hline
  6 & 147774    & 232748   & 170378 (73\%)/ 31668 (13\%) & 0.99 / 0.68 \\ \hline
  7 & 360354    & 601462   & 453789 (75\%)/ 66438 (11\%) & 0.99 / 0.74 \\ \hline
\end{tabular}


\section{Распределенное завершение}
\label{sec:distributed-termination}


\end{document}

