\documentclass[a4paper,12pt,notitlepage]{article}

\usepackage{ucs}
\usepackage[utf8x]{inputenc}
\usepackage[T1]{fontenc}
\usepackage[english,russian]{babel}

\usepackage{textcomp}
\usepackage{indentfirst}
\usepackage{verbatim}

\sloppy

\begin{document}

\section{Параллельная верификация CTL-формул}
\label{sec:ctl-verification}

\paragraph{Актуальность.}
Проверка модели -- автоматический формальный подход, в ходе которого проверяется
соответствие дискретной детерминированной модели программного или программно-аппаратного
комплекса некоторой спецификации. Проверка модели позволяет находить ряд тяжело
обнаруживаемых ошибок на стадии проектирования.

Наиболее часто используемой нотацией для задания спецификаций являются временные логики:
логика линейного времени LTL, логика ветвящегося времени CTL, логика расширенного времени
ETL, и т.д. Они обладают разной, плохо сравнимой степенью выразительности и различными
областями применения. К примеру, логика линейного времени хорошо подходит для утверждений
об отсутствии блокировок и ресурного голодания в программных алгоритмах, а логика
ветвящегося времени -- для утверждений об обработке исключительных ситуаций в аппаратных
средствах.

Проверка моделей является ресурсоемким процессом (задача класса сложности PSPACE),
требующим как больших объемов ОЗУ, так и процессорного времени, поэтому актуальной
является задача распараллеливания ее между различными узлами вычислительной сети. Задача
проверки формул временной логики для графа состояний, распределенного между различными
узлами, является достаточно сложной и не имеющей на данный момент окончательного
решения. 

\paragraph{Научная новизна.}
Для иных формализмов, таких, как логика ветвящегося времени или мю-исчисление, работы по
параллельной верификации на данный момент отсутствуют. В тоже время CTL, к примеру,
используется в средства верификации компании Intel.

\paragraph{Текущие работы по теме.}
Работы по созданию параллельных средств для проверки моделей ведутся еще с 99 года, однако
практические результаты пока получены лишь в университете Брно доктором Жири Барнатом. В
его работах рассматривается одна из временных логик -- логика линейного времени (LTL). 

\section{Семантический анализ код с целью установления авторства}
\label{sec:saa}

\paragraph{Актуальность.}
С ростом числа преступлений, связанных с ПО, возникла необходимость в области, называемоей
software forensics (<<судебный анализ исходного кода>>) и технологиях для семантического
анализа исходных текстов. В данной области выделяются следующие подзадачи:
\begin{enumerate}
\item различение авторов (определить, одним или несколькими людьми написан данный отрывок кода);
\item идентификация автора (определить, написан ли данный отрывок кода данным автором на
  основе его предыдущих кодов);
\item характеризация автора (оценить те или иные характеристики автора на основе отрывка
  кода; например, какого рода образование по специальности он скорее всего получил и т.п.);
\item определение намерений автора (определить, случайна или намерена данная ошибка в коде);
\item обнаружение плагиата (определить, написан ли один программный продукт на основе кода
  другого).
\end{enumerate}

\paragraph{Научная новизна.}
Данная область довольно обширна как по задачам, так и по методам их решения: предлагается
множество методов, основанных параметрическом анализе (по десяткам различных параметров
исходных текстов), нейронных сетях, CBR (case-based reasoning) и т.д. Полных и законченных
решений на данный момент нет.

\paragraph{Текущие работы по теме.}
Ряд работ в области software forensics ведется в течение последних 10--15 лет в
университетах по всему миру, включая Laboratory of Information and Communication Systems
Security (Греция) и Software Metrics Research Laboratory (Новая Зеландия). В последнем
создана системы анализа исходных текстов IDENTIFIED.

\section{Оптимизация программ за счет использования разработки, основывающейся на
  структурах данных (Data-Oriented Design)}
\label{sec:data-driven-design}

\paragraph{Актуальность.}
С ростом разрыва в скорости работы ЦПУ и ОЗУ все более актуальной становится проблема
правильного упорядочивания данных в памяти программы, которое приводило бы к меньшему
числу кэш-промахов и большей скорости работы. В наше время ведущим подходом к разработке
является объектно-ориентированный подход (Object-Oriented Design), использование которого
приводит к тенденции <<размазывания>> однотипных данных по памяти программы, что при
массовой обработке таких данных часто приводит к ухудшению производительности по ряду
причин, в основном из-за роста числа кэш-промахов. 

Альтернативой является Data-Oriented Design (основывающийся на структурах данных вместо
концептуальных сущностей), при которых однотипные простые данные (например, поле объекта)
хранятся совсместно (а <<объекты>> при этом размазываются). Такой подход широко
применяется в определенных областях разработки ПО, например в СУБД и в играх, однако не
является традиционных, поскольку человеку проще мыслить в терминах объектов предметной
области, чем оптимального расположения данных в памяти.

Исходя из этого, предлагается разработка метода для автоматической оптимизации программ
путем переупорядочивания расположения данных в памяти программы так, чтобы уменьшить число
кэш-промахов при работе с ними.

\paragraph{Научная новизна.}
Такой метод позволит получить прирост в производительности программы за счет лучшего
расположения данных в сочетании с использованием традиционного подхода к разработке ПО.

\paragraph{Текущие работы по теме.}
Сам по себе Data-Oriented Design не нов, широко описан и используется в упомянутых
категориях ПО, однако работ по разработке автоматического метода оптимизации на основе
этого подхода пока что не существует.

\end{document}

%%% Local Variables: 
%%% mode: latex
%%% TeX-master: t
%%% End: 
